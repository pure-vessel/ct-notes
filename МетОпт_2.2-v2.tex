\documentclass{article}
\usepackage{ifluatex}
\ifluatex 
    \usepackage{fontspec}
    \setsansfont[
        Path = C:/Windows/Fonts/,
        Extension = .ttf ,
        BoldFont={cmunsx} ,
        ItalicFont={cmunsi} ,
        BoldItalicFont={cmunso} ,
        UprightFont={cmunss}
    ]{CMU Sans Serif}
    \setmainfont[
        Path = C:/Windows/Fonts/,
        Extension = .ttf ,
        BoldItalicFont={cmunbi} ,
        ItalicFont={cmunti} ,
        BoldFont={cmunbx} ,
        UprightFont={cmunrm}
    ]{CMU Serif}
    \setmonofont[
        Path = C:/Windows/Fonts/,
        Extension = .ttf ,
        % LightFont={cmunbtl} ,
        BoldItalicFont={cmuntx} ,
        % LightItalicFont={cmunbto} ,
        BoldFont={cmuntb} ,
        ItalicFont={cmunit} ,
        UprightFont={cmuntt}
    ]{CMU Typewriter Text}
    \defaultfontfeatures{Ligatures={TeX}}
\else
    \usepackage[T2A]{fontenc}
    \usepackage[utf8]{inputenc}
\fi
\usepackage[english,russian]{babel}
\usepackage{anyfontsize}
\usepackage{amssymb,latexsym,amsmath,amscd,mathtools,wasysym,stmaryrd}
\usepackage[shortlabels]{enumitem}
\usepackage[makeroom]{cancel}
\usepackage{graphicx}
\usepackage{geometry}
\usepackage{verbatim}
\usepackage{fvextra}

\usepackage{longtable}
\usepackage{multirow}
\usepackage{multicol}
\usepackage{tabu}
\usepackage{arydshln} % \hdashline and :
\usepackage{makecell} % \makecell for line breaks
\usepackage{tabularx}
\usepackage{xltabular}
\renewcommand\tabularxcolumn[1]{m{#1}} % for vertical centering text in X column


\usepackage{float}
\makeatletter
\g@addto@macro\@floatboxreset\centering
\makeatother
\setlength{\parindent}{0pt}
\usepackage{caption}
\usepackage{csquotes}
\usepackage[bb=dsserif]{mathalpha}
\usepackage[normalem]{ulem}

\usepackage[e]{esvect}
\let\vec\vv

\usepackage[x11names]{xcolor}
\colorlet{darkgreen}{black!25!blue!50!green}


%% Here f*cking with mathabx
\DeclareFontFamily{U}{matha}{\hyphenchar\font45}
\DeclareFontShape{U}{matha}{m}{n}{
    <5> <6> <7> <8> <9> <10> gen * matha
    <10.95> matha10 <12> <14.4> <17.28> <20.74> <24.88> matha12
}{}
\DeclareSymbolFont{matha}{U}{matha}{m}{n}
\DeclareFontFamily{U}{mathb}{\hyphenchar\font45}
\DeclareFontShape{U}{mathb}{m}{n}{
    <5> <6> <7> <8> <9> <10> gen * mathb
    <10.95> matha10 <12> <14.4> <17.28> <20.74> <24.88> mathb12
}{}
\DeclareSymbolFont{mathb}{U}{mathb}{m}{n}

\DeclareMathSymbol{\tmp}{\mathrel}{mathb}{"15}
\let\defeq\tmp
\DeclareMathSymbol{\tmp}{\mathrel}{mathb}{"16}
\let\eqdef\tmp

\usepackage{trimclip}
\DeclareMathOperator{\updownarrows}{\clipbox{0pt 0pt 4.175pt 0pt}{$\upuparrows$}\hspace{-.825px}\clipbox{0pt 0pt 4.175pt 0pt}{$\downdownarrows$}}
\DeclareMathOperator{\downuparrows}{\clipbox{0pt 0pt 4.175pt 0pt}{$\downdownarrows$}\hspace{-.825px}\clipbox{0pt 0pt 4.175pt 0pt}{$\upuparrows$}}

\makeatletter
\providecommand*\deletecounter[1]{%
    \expandafter\let\csname c@#1\endcsname\@undefined}
\makeatother


\usepackage{hyperref}
\hypersetup{
    %hidelinks,
    colorlinks=true,
    linkcolor=darkgreen,
    urlcolor=blue,
    breaklinks=true,
}

\usepackage{pgf}
\usepackage{pgfplots}
\pgfplotsset{compat=newest}
\usepackage{tikz,tikz-3dplot}
\usepackage{tkz-euclide}
\usetikzlibrary{calc,automata,patterns,angles,quotes,backgrounds,shapes.geometric,trees,positioning,decorations.pathreplacing}
\usetikzlibrary{lindenmayersystems}
\pgfkeys{/pgf/plot/gnuplot call={cd Output && gnuplot}}
\usepgfplotslibrary{fillbetween,polar}
\usetikzlibrary{quotes,babel}
\ifluatex
\usetikzlibrary{graphs,graphs.standard,graphdrawing}
\usegdlibrary{layered,trees,circular,force}
\else
% \errmessage{Run with LuaTeX, if you want to use gdlibraries}
\fi
\makeatletter
\newcommand\currentnode{\the\tikz@lastxsaved,\the\tikz@lastysaved}
\makeatother

%\usepgfplotslibrary{external} 
%\tikzexternalize

\makeatletter
\newcommand*\circled[2][1.0]{\tikz[baseline=(char.base)]{
        \node[shape=circle, draw, inner sep=2pt,
        minimum height={\f@size*#1},] (char) {#2};}}
\makeatother

\newcommand{\existence}{{\circled{$\exists$}}}
\newcommand{\uniqueness}{{\circled{$\hspace{0.5px}!$}}}
\newcommand{\rightimp}{{\circled{$\Rightarrow$}}}
\newcommand{\leftimp}{{\circled{$\Leftarrow$}}}

\DeclareMathOperator{\sign}{sign}
\DeclareMathOperator{\Cl}{Cl}
\DeclareMathOperator{\proj}{pr}
\DeclareMathOperator{\Arg}{Arg}
\DeclareMathOperator{\supp}{supp}
\DeclareMathOperator{\diag}{diag}
\DeclareMathOperator{\tr}{tr}
\DeclareMathOperator{\rank}{rank}
\DeclareMathOperator{\Lat}{Lat}
\DeclareMathOperator{\Lin}{Lin}
\DeclareMathOperator{\Ln}{Ln}
\DeclareMathOperator{\Orbit}{Orbit}
\DeclareMathOperator{\St}{St}
\DeclareMathOperator{\Seq}{Seq}
\DeclareMathOperator{\PSet}{PSet}
\DeclareMathOperator{\MSet}{MSet}
\DeclareMathOperator{\Cyc}{Cyc}
\DeclareMathOperator{\Hom}{Hom}
\DeclareMathOperator{\End}{End}
\DeclareMathOperator{\Aut}{Aut}
\DeclareMathOperator{\Ker}{Ker}
\DeclareMathOperator{\Def}{def}
\DeclareMathOperator{\Alt}{Alt}
\DeclareMathOperator{\Sim}{Sim}
\DeclareMathOperator{\Int}{Int}
\DeclareMathOperator{\grad}{grad}
\DeclareMathOperator{\sech}{sech}
\DeclareMathOperator{\csch}{csch}
\DeclareMathOperator{\asin}{\sin^{-1}}
\DeclareMathOperator{\acos}{\cos^{-1}}
\DeclareMathOperator{\atan}{\tan^{-1}}
\DeclareMathOperator{\acot}{\cot^{-1}}
\DeclareMathOperator{\asec}{\sec^{-1}}
\DeclareMathOperator{\acsc}{\csc^{-1}}
\DeclareMathOperator{\asinh}{\sinh^{-1}}
\DeclareMathOperator{\acosh}{\cosh^{-1}}
\DeclareMathOperator{\atanh}{\tanh^{-1}}
\DeclareMathOperator{\acoth}{\coth^{-1}}
\DeclareMathOperator{\asech}{\sech^{-1}}
\DeclareMathOperator{\acsch}{\csch^{-1}}

\newcommand*{\scriptA}{{\mathcal{A}}}
\newcommand*{\scriptB}{{\mathcal{B}}}
\newcommand*{\scriptC}{{\mathcal{C}}}
\newcommand*{\scriptD}{{\mathcal{D}}}
\newcommand*{\scriptF}{{\mathcal{F}}}
\newcommand*{\scriptG}{{\mathcal{G}}}
\newcommand*{\scriptH}{{\mathcal{H}}}
\newcommand*{\scriptK}{{\mathcal{K}}}
\newcommand*{\scriptL}{{\mathcal{L}}}
\newcommand*{\scriptM}{{\mathcal{M}}}
\newcommand*{\scriptP}{{\mathcal{P}}}
\newcommand*{\scriptQ}{{\mathcal{Q}}}
\newcommand*{\scriptR}{{\mathcal{R}}}
\newcommand*{\scriptT}{{\mathcal{T}}}
\newcommand*{\scriptU}{{\mathcal{U}}}
\newcommand*{\scriptX}{{\mathcal{X}}}
\newcommand*{\Cnk}[2]{\left(\begin{matrix}#1\\#2\end{matrix}\right)}
\newcommand*{\im}{{\mathbf i}}
\newcommand*{\id}{{\mathrm{id}}}
\newcommand*{\compl}{^\complement}
\newcommand*{\dotprod}[2]{{\left\langle{#1},{#2}\right\rangle}}
\newcommand\matr[1]{\left(\begin{matrix}#1\end{matrix}\right)}
\newcommand\matrd[1]{\left|\begin{matrix}#1\end{matrix}\right|}
\newcommand\arr[2]{\left(\begin{array}{#1}#2\end{array}\right)}

\DeclareMathOperator{\divby}{\scalebox{1}[.65]{\vdots}}
\DeclareMathOperator{\toto}{\rightrightarrows}
\DeclareMathOperator{\ntoto}{\not\rightrightarrows}

\newcommand{\bigmid}{\mathrel{\big|}}
\newcommand{\Bigmid}{\mathrel{\Big|}}
\newcommand{\biggmid}{\mathrel{\bigg|}}

\newcommand{\undercolorblack}[2]{{\color{#1}\underline{\color{black}#2}}}
\newcommand{\undercolor}[2]{{\colorlet{tmp}{.}\color{#1}\underline{\color{tmp}#2}}}

\usepackage{adjustbox}

\geometry{margin=1in}
\usepackage{fancyhdr}
\pagestyle{fancy}
\fancyfoot[L]{}
\fancyfoot[C]{Иванов Тимофей}
\fancyfoot[R]{\pagename\ \thepage}
\fancyhead[L]{}
\fancyhead[R]{\leftmark}
\renewcommand{\sectionmark}[1]{\markboth{#1}{}}

\setcounter{tocdepth}{5}
\usepackage{amsthm}
\usepackage{chngcntr}

\theoremstyle{definition}
\newtheorem{definition}{Определение}
\counterwithin*{definition}{section}

\theoremstyle{plain}
\newtheorem{theorem}{Теорема}
\counterwithin*{theorem}{section} % Without changing appearance
\newtheorem{lemma}{Лемма}
\counterwithin*{lemma}{section}
\newtheorem{corollary}{Следствие}[theorem]
\counterwithin{corollary}{theorem} % Changing appearance
\counterwithin{corollary}{lemma}
\newtheorem*{claim}{Утверждение}
\newtheorem{property}{Свойство}[definition]

\theoremstyle{remark}
\newtheorem*{remark}{Замечание}
\newtheorem*{example}{Пример}

\newtheorem*{discourse}{Рассуждение}


%\renewcommand\qedsymbol{$\blacksquare$}

\counterwithin{equation}{section}


\fancyhead[L]{Методы оптимизации}

\begin{document}
    \noindent Оптимизация~--- это о чём? Ну, о том, что мы ищем минимум (или максимум) какой-то функции на каком-то множестве. Причём минимум ищется обычно локальный, потому что глобальный искать очень сложно.\\
    Рассмотрим непрерывную оптимизацию: наша функция непрерывна. В таких методах применяется принцип чёрного ящика: мы не анализируем то, что делает функция, нас интересуют только её значения. Иногда нас также интересует градиент функции или гессиан. В зависимости от того, производная какого порядка нам нужно в нашем методе, говорят о том, что наш метод~--- это метод соответствующего порядка. Методы второго порядка~--- это, например, методы Ньютона. Также есть методы квази-Ньютона, которые пытаются как-то аппроксимировать гессиан, нь сейчас не о них.\\
    Зачем нам вообще градиент и гессиан? Потому что они позволяют разложить функцию в ряд Тейлора, и выглядеть он будет так:
    \[
    f(x+\Delta x)=f(x)+\Delta x^T\nabla f(x)+\frac12\Delta x^TH_f(x)\Delta x+o(\|\Delta x\|^2)
    \]
    Более высокие порядки не используются обычно, потому что там возникают такие операции как взятие обратной матрицы, которые вводят численную нестабильность.
    \begin{example}
        Тернарный поиск. Все мы его знаем. Это метод нулевого порядка. Берём функцию, которая сначала убывает, потом возрастает и итеративно делим отрезок на три части, на каждом этапе отбрасывая одну треть.\\
        Можно немного упростить жизнь себе, использовав метод золотого сечения. А именно отрезок $[a;b]$ мы делим на три части точками $x_1=b-\frac{b-a}\Phi$, $x_2=a+\frac{b-a}\Phi$, где $\Phi=\frac{1+\sqrt5}2$. Остальное как в обычном тернарном поиске. При таком разделении значение в одной из двух точек будет переиспользовано, ведь $x_1$ делит $[a;x_2]$ в соотношении золотого сечения и $x_2$ делит $[x_1;b]$ в соотношении золотого сечения.
    \end{example}\noindent
    Нулевого порядка в целом больше ничего не придумать.\\
    Для методов первого порядка обычно итеративно выбирают последовательность точек, значения в которых должны уменьшаться. Как выбирать~--- рассмотрим позже. А сейчас подумаем, когда останавливаться.
    \begin{definition}
        Пусть $\{x_k\}$~--- последовательность точек, которые выбирает метод, а $x^*$~--- локальный минимум, к которому метод стремится. Тогда если выполнено
        \[
        \lim\limits_{k\to\infty}\frac{\|x_{k+1}-x^*\|}{\|x_k-x^*\|^\mu}<r
        \]
        то $r$ называют \textbf{скоростью сходимости}, а $\mu$~--- \textbf{степенью сходимости}.
    \end{definition}
    \begin{definition}
        \textbf{Контурными линиями} называются множества
        \[
        L_f(a)=\{x\mid f(x)=a\}
        \]
    \end{definition}
    \begin{claim}
        Контурные линии ортогональны градиенту функции.
    \end{claim}
    \paragraph{Условия оптимальности.}
    Хочется найти какие-нибудь критерии оптимальности точки в какой-то окрестности. У нас могут быть достаточные условия, могу быть необходимые, и все они следуют из квадратичной аппроксимации функции.\\
    Ну, точка оптимальна, если при добавлении произвольного $\Delta x$ функция возрастёт. Заметим, что $\nabla f(x)$ и $H_f(x)$ никак не зависят от $x$. Для методов первого порядка есть только необходимое условие:
    \begin{theorem}
        Если точка оптимальна, то $\nabla f(x)=0$.
    \end{theorem}
    \begin{proof}
        Ну, если он не ноль, то можно найти достаточно маленькое $\Delta x$, противонаправленное $\nabla f(x)$, и функция уменьшится. А значит точка не оптимальна была.
    \end{proof}\noindent
    Самое интересное, что для хороших функций можно решить уравнение $\nabla f(x)=0$ и получить аналитическое решение. Например, так отлично решается линейная регрессия.\\
    А вот если у нас есть гессиан, то мы и достаточное условие можем сформулировать. Это условие доказывается через спектральное разложение гессиана.
    \begin{theorem}
        Если все собственные числа гессиана в точке положительные, то эта точка~--- минимум.\\
        Если точка~--- минимум, то все собственные числа гессиана неотрицательны.
    \end{theorem}
    \begin{proof}
        \[\begin{split}
            f(x+\Delta x)&\approx f(x)+\frac12\Delta x^TH_f(x)\Delta x=\\
            &=f(x)+\frac12\Delta x^TQ\land Q^T\Delta x=\\
            &=f(x)+\frac12(Q^T\Delta x)^T\land Q^T\Delta x=\\
            &=f(x)+\frac12\sum\limits_i\lambda_i\|(Q^T\Delta x)_i\|\\
        \end{split}\]
    \end{proof}\noindent
    \paragraph{Линейные ограничения.}
    Давайте добавим какие-нибудь ограничения. например, линейные ограничения: минимизировать функцию $f$ при условии $Ax=b$. Тогда $A\Delta x=0$. Соотвественно, проверяя условия необходимости и/или достаточности, нам надо проверять только те направления, для которых верно $A\Delta x=0$. Как с этим работать? Ну, довольно легко: надо рассмотреть базис $\ker A$ (назовём $N$ матрицу проекции на $\ker A$). И все $\Delta x$ надо заменить на $N\Delta x$. Тогда наш $\Delta x$ всегда будет правильным. Что будет с нашим рядом Тейлора?
    \[
    f(x+N\Delta x)=f(x)+\Delta x^TN^T\nabla f(x)+\frac12\Delta x^TN^TH_f(x)N\Delta x+o(\|\Delta x\|^2)
    \]
    В итоге можно только домножать градиент и гессиан на $N$.
\end{document}