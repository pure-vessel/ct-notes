\newcommand{\convexpath}[2]{
    [   
    create hullcoords/.code={
        \global\edef\namelist{#1}
        \foreach [count=\counter] \nodename in \namelist {
            \global\edef\numberofnodes{\counter}
            \coordinate (hullcoord\counter) at (\nodename);
        }
        \coordinate (hullcoord0) at (hullcoord\numberofnodes);
        \pgfmathtruncatemacro\lastnumber{\numberofnodes+1}
        \coordinate (hullcoord\lastnumber) at (hullcoord1);
    },
    create hullcoords
    ]
    ($(hullcoord1)!#2!-90:(hullcoord0)$)
    \foreach [
    evaluate=\currentnode as \previousnode using \currentnode-1,
    evaluate=\currentnode as \nextnode using \currentnode+1
    ] \currentnode in {1,...,\numberofnodes} {
        let \p1 = ($(hullcoord\currentnode) - (hullcoord\previousnode)$),
        \n1 = {atan2(\y1,\x1) + 90},
        \p2 = ($(hullcoord\nextnode) - (hullcoord\currentnode)$),
        \n2 = {atan2(\y2,\x2) + 90},
        \n{delta} = {Mod(\n2-\n1,360) - 360}
        in 
        {arc [start angle=\n1, delta angle=\n{delta}, radius=#2]}
        -- ($(hullcoord\nextnode)!#2!-90:(hullcoord\currentnode)$) 
    }
} % Nodes CW. Supposes tikz >= 3.0, else swap atan2 arguments.


\tikzset{treenode/.style={
        inner sep=0pt,
        text width=5mm,
        scale=.8,
        circle,
        outer sep=2mm,
        align=center,
        draw=black,
        fill=white,
        thin
}}
\tikzset{subtree/.style={
        regular polygon,
        regular polygon sides=3,
        inner sep=0pt,
        text width=5mm,
        yscale=1.5,
        xscale=.7,
        draw=black,
        fill=white,
        thin
}}
\tikzset{
    ->-/.style={
        decoration={
            markings,
            mark=at position #1 with {\arrow{>}}
        },
        postaction={decorate}
    },
    ->-/.default=0.5,
    -<-/.style={
        decoration={
            markings,
            mark=at position #1 with {\arrow{<}}
        },
        postaction={decorate}
    },
    -<-/.default=0.5
}
